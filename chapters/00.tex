\chapter{Abstract}

Ziel des Projektes ist die Entwicklung einer innovativen Finanzmanagementplattform, die es einzelnen Nutzern ermöglicht, ihre finanziellen Angelegenheiten effizienter
zu verwalten und finanzielle Ziele zu erreichen. Kernfunktionen sind die Verfolgung von Einnahmen und Ausgaben für einen transparenten Überblick über die finanzielle Situation, KI-gestützte Budgetplanung und -analyse zur optimierten Ausgabenkontrolle, Spar- und Anlagemanagement zur Förderung intelligenter Finanzentscheidungen und die Möglichkeit, finanzielle Ziele zu setzen und deren Erreichung in Echtzeit zu verfolgen.

Die rechtliche Prüfung des Projekts erfordert eine detaillierte Analyse verschiedener Aspekte. Dazu gehört die Ermittlung der personenbezogenen Daten einschließlich der sensiblen Daten sowie die Beschreibung der konkreten Verarbeitungsvorgänge. Es ist wichtig, die Rechtsgrundlagen für jede Verarbeitungstätigkeit zu prüfen und sicherzustellen, dass diese auf einer gültigen Rechtsgrundlage beruhen, sei es durch die Einwilligung der betroffenen Personen oder durch eine Interessenabwägung. Darüber hinaus muss der für die Verarbeitung Verantwortliche eindeutig bestimmt werden und es muss geprüft werden, ob Auftragsverarbeiter beteiligt sind. Des Weiteren sind die bereits getroffenen technisch-organisatorischen Maßnahmen zur Einhaltung der Datenschutzgrundsätze auf ihre Angemessenheit hin zu überprüfen und gegebenenfalls ergänzende Maßnahmen zu treffen. Schließlich ist es wichtig, die wesentlichen Informationen, die den betroffenen Personen über die Datenverarbeitung mitgeteilt werden müssen, klar zu definieren, um Transparenz und die Einhaltung der Datenschutzvorschriften zu gewährleisten.
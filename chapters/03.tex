\chapter{Rechtsgrundlagen für die Verarbeitungsvorgänge auf der Plattform}

Um die Verarbeitungsvorgänge gemäß der in der Anfrage beschriebenen Aktivitäten auf einer rechtlichen Basis nach der DS-GVO zu rechtfertigen, können wir auf Artikel 6 der DS-GVO zurückgreifen. Hier ist, wie jeder Verarbeitungsvorgang gerechtfertigt werden kann:

\begin{enumerate}
    \item \textbf{Datenerhebung:} Dies kann auf Basis der Einwilligung der betroffenen Person gemäß Art. 6 Abs. 1 a DS-GVO gerechtfertigt werden, da die Nutzer aktiv ihre persönlichen und finanziellen Informationen bereitstellen. Darüber hinaus kann die Erfassung finanzieller Daten für die Vertragsdurchführung (Nutzung der Plattformdienste) erforderlich sein, was eine Rechtfertigung nach Art. 6 Abs. 1 b DS-GVO darstellt.

    \item \textbf{Speicherung der Daten:} Die Speicherung von Daten kann aufgrund der Notwendigkeit für die Erfüllung eines Vertrags (Art. 6 Abs. 1 b DS-GVO) gerechtfertigt werden, insbesondere um den Nutzern die Nutzung der Plattformfunktionen zu ermöglichen.

    \item \textbf{Analyse der Daten:} Die Nutzung von KI zur Budgetplanung und -analyse fällt unter die Kategorie der Vertragsdurchführung (Art. 6 Abs. 1 b DS-GVO), da sie direkt mit den angebotenen Dienstleistungen der Plattform zusammenhängt.

    \item \textbf{Änderung der Daten:} Das Recht der Nutzer, ihre Daten zu aktualisieren oder zu korrigieren, kann als Teil der Vertragserfüllung (Art. 6 Abs. 1 b DS-GVO) gesehen werden, sowie durch die Einwilligung der betroffenen Person (Art. 6 Abs. 1 a DS-GVO).

    \item \textbf{Löschung von Daten:} Die Löschung von Daten auf Anfrage der Nutzer kann auf die Einwilligung gemäß Art. 6 Abs. 1 a DS-GVO gestützt werden, insbesondere wenn die Datenverarbeitung nicht mehr notwendig ist oder wenn der Nutzer seine Einwilligung widerruft
\end{enumerate}

In jedem Fall ist es wichtig, dass die Einwilligung der betroffenen Person spezifisch, informiert und eindeutig ist, um die Rechtmäßigkeit der Datenverarbeitung zu gewährleisten. Bei einer Verarbeitung aufgrund berechtigter Interessen muss außerdem eine sorgfältige Interessenabwägung vorgenommen werden, um sicherzustellen, dass die Rechte und Freiheiten der betroffenen Person nicht überwiegen.
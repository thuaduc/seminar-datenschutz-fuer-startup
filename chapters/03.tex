\chapter{Rechtsgrundlagen für die Verarbeitungsvorgänge auf der Plattform}

Für die Verarbeitungsvorgänge personenbezogener Daten in der Finanzmanagement-Plattform können verschiedene Rechtsgrundlagen gemäß der Allgemeinen Datenschutzverordnung (DSGVO) oder entsprechenden lokalen Datenschutzgesetzen herangezogen werden. Hier sind Beispiele für Rechtsgrundlagen für die genannten Verarbeitungsvorgänge:

\begin{enumerate}
    \item \textbf{Einwilligung (Art. 6 Abs. 1 lit. a DSGVO):}
        Die Nutzer geben ihre ausdrückliche Einwilligung zur Verarbeitung ihrer persönlichen und finanziellen Daten bei der Registrierung auf der Plattform und beim manuellen Eingeben oder Importieren ihrer Finanzdaten. Die Einwilligung umfasst spezifische Verarbeitungsvorgänge wie die Analyse finanzieller Daten für personalisierte Empfehlungen. Nutzer haben das Recht, ihre Einwilligung jederzeit zu widerrufen.

    \item \textbf{Vertragserfüllung (Art. 6 Abs. 1 lit. b DSGVO):}
        Die Verarbeitung personenbezogener Daten ist notwendig, um den mit den Nutzern geschlossenen Vertrag über die Nutzung der Finanzmanagement-Plattform zu erfüllen. Dies beinhaltet die Verarbeitung von Daten zur Kontoerstellung, Budgetplanung, Spar- und Investitionsmanagement sowie zum Schuldenabbau.

    \item \textbf{Erfüllung rechtlicher Verpflichtungen (Art. 6 Abs. 1 lit. c DSGVO):}
        In einigen Fällen kann die Verarbeitung personenbezogener Daten erforderlich sein, um rechtlichen Verpflichtungen nachzukommen, z.B. bei gesetzlich vorgeschriebenen Finanzberichten oder im Rahmen von Betrugsbekämpfungsmaßnahmen.

    \item \textbf{Schutz lebenswichtiger Interessen (Art. 6 Abs. 1 lit. d DSGVO):}
        Dieser Grund ist weniger relevant für die Finanzmanagement-Plattform, könnte aber in bestimmten Szenarien zur Anwendung kommen, z.B. wenn die Verarbeitung personenbezogener Daten notwendig ist, um lebenswichtige Interessen der Nutzer oder einer anderen natürlichen Person zu schützen.
\end{enumerate}


Jeder dieser Verarbeitungsvorgänge erfordert eine spezifische Rechtsgrundlage, die je nach Kontext der Datenverarbeitung und der Art der personenbezogenen Daten ausgewählt wird. Wichtig ist, dass die Plattform die entsprechende Rechtsgrundlage klar dokumentiert und den Nutzern transparent kommuniziert, auf welcher Basis ihre Daten verarbeitet werden.

\chapter{Wesentliche Informationspflichten nach DSGVO für Nutzer der Finanzmanagement-Plattform}

Gemäß der Datenschutz-Grundverordnung (DSGVO) sind Sie verpflichtet, den betroffenen Personen eine Reihe von wesentlichen Informationen bezüglich der Verarbeitung ihrer personenbezogenen Daten mitzuteilen. Im Kontext des Projekts zur Entwicklung einer Plattform für das persönliche Finanzmanagement sollten folgende Informationen den Nutzern klar und verständlich kommuniziert werden:

\begin{enumerate}

    \item \textbf{Identität und Kontaktdaten des Verantwortlichen:}
        Den Nutzern muss mitgeteilt werden, wer für die Verarbeitung ihrer personenbezogenen Daten verantwortlich ist. Dies umfasst den Namen und die Kontaktdaten des Unternehmens oder der Organisation sowie gegebenenfalls des Datenschutzbeauftragten.

    \item \textbf{Zwecke der Datenverarbeitung:}
        Es muss klar angegeben werden, für welche spezifischen Zwecke die personenbezogenen Daten verarbeitet werden (z.B. Budgetverwaltung, Spar- und Investitionsplanung).

    \item \textbf{Rechtsgrundlage der Verarbeitung:}
        Die Rechtsgrundlage für die Verarbeitung der personenbezogenen Daten muss den Nutzern mitgeteilt werden (z.B. Verarbeitung aufgrund der Einwilligung der Nutzer, zur Vertragserfüllung oder aufgrund eines berechtigten Interesses).

    \item \textbf{Empfänger oder Kategorien von Empfängern der personenbezogenen Daten:} 
        Falls die Daten an Dritte weitergegeben werden, sollten die Nutzer darüber informiert werden, wer diese Empfänger sind oder welche Kategorien von Empfängern existieren (z.B. Auftragsverarbeiter, Finanzdienstleister).

    \item \textbf{Datenübermittlung an ein Drittland oder eine internationale Organisation:} 
        Wenn Daten außerhalb der EU/EWR übermittelt werden, muss dies angegeben werden, zusammen mit den Sicherheitsmaßnahmen, die zum Schutz der Daten ergriffen wurden.

    \item \textbf{Dauer der Speicherung der personenbezogenen Daten:} 
        Die Nutzer müssen darüber informiert werden, wie lange ihre Daten gespeichert werden oder welche Kriterien für die Festlegung dieser Dauer angewendet werden.

    \item \textbf{Rechte der betroffenen Personen:} 
        Den Nutzern müssen ihre Rechte in Bezug auf ihre Daten mitgeteilt werden, einschließlich des Rechts auf Zugang, Berichtigung, Löschung („Recht auf Vergessenwerden“), Einschränkung der Verarbeitung, Datenübertragbarkeit und Widerspruch gegen die Verarbeitung.

    \item \textbf{Das Recht auf Widerruf der Einwilligung:} 
        Wenn die Verarbeitung auf der Einwilligung basiert, sollte den Nutzern mitgeteilt werden, dass sie das Recht haben, ihre Einwilligung jederzeit zu widerrufen, ohne dass dies die Rechtmäßigkeit der aufgrund der Einwilligung bis zum Widerruf erfolgten Verarbeitung berührt.

    \item \textbf{Das Recht, eine Beschwerde bei einer Aufsichtsbehörde einzureichen:} 
        Die Nutzer sollten darauf hingewiesen werden, dass sie das Recht haben, eine Beschwerde bei einer Datenschutzaufsichtsbehörde einzureichen, wenn sie der Meinung sind, dass die Verarbeitung ihrer personenbezogenen Daten gegen die DSGVO verstößt.
\end{enumerate}

Diese Informationen sollten den Nutzern zum Zeitpunkt der Erhebung ihrer personenbezogenen Daten in einer klaren, transparenten und leicht zugänglichen Form zur Verfügung gestellt werden.
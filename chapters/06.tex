\chapter{Wesentliche Informationspflichten nach DS-GVO für Nutzer der Plattform}

Im Kontext des Projekts sollten folgende Informationen den Nutzern klar und verständlich kommuniziert werden:

\begin{enumerate}

    \item \textbf{Verantwortlicher und Datenschutzbeauftragter (Art. 13 Abs. 1 a-b DS-GVO):} Name und Kontaktdaten des Verantwortlichen (und gegebenenfalls seines Vertreters) sowie die Kontaktdaten des Datenschutzbeauftragten.

    \item \textbf{Zwecke und Rechtsgrundlage der Verarbeitung (Art. 13 Abs. 1 c, Art. 14 Abs. 1 c DS-GVO):} Die spezifischen Zwecke, für die die personenbezogenen Daten verarbeitet werden sollen, und die Rechtsgrundlage der Verarbeitung müssen klar definiert werden.

    \item \textbf{Berechtigte Interessen (Art. 13 Abs. 1 d DS-GVO):} Falls die Verarbeitung auf berechtigten Interessen beruht, sind diese dem Betroffenen mitzuteilen.

    \item \textbf{Empfänger oder Kategorien von Empfängern (Art. 13 Abs. 1 e DS-GVO):} Die Identität von Empfängern oder die Kategorien von Empfängern, denen die Daten offengelegt wurden oder werden.

    \item \textbf{Datenübermittlung (Art. 13 Abs. 1 f DS-GVO):} Informationen über die Übermittlung personenbezogener Daten an ein Drittland oder eine internationale Organisation sowie die Vorkehrungen zum Schutz der Daten.

    \item \textbf{Speicherdauer (Art. 13 Abs. 2 a DS-GVO):} Die geplante Dauer der Speicherung der personenbezogenen Daten oder die Kriterien zur Festlegung dieser Dauer.

    \item \textbf{Rechte der betroffenen Person (Art. 13 Abs. 2 b DS-GVO):} Das Recht auf Zugang, Berichtigung, Löschung der Daten, Einschränkung der Verarbeitung, Widerspruch gegen die Verarbeitung, und das Recht auf Datenübertragbarkeit.

    \item \textbf{Widerrufsrecht (Art. 13 Abs. 2 c DS-GVO):} Das Recht, eine Einwilligung jederzeit zu widerrufen, ohne dass die Rechtmäßigkeit der Verarbeitung, die auf der Einwilligung vor ihrem Widerruf beruht, beeinträchtigt wird.

    \item \textbf{Beschwerderecht (Art. 13 Abs. 2 d DS-GVO):} Das Recht, Beschwerde bei einer Aufsichtsbehörde einzulegen.
\end{enumerate}

Diese Informationen sollten den Nutzern zum Zeitpunkt der Erhebung ihrer personenbezogenen Daten in einer klaren, transparenten und leicht zugänglichen Form zur Verfügung gestellt werden.
\chapter{Implementierte Datenschutzmaßnahmen und deren Bewertung für die Plattform}

Basierend auf den Anforderungen der Art. 5 Abs. 1 f DS-GVO kann eine Reihe von Maßnahmen identifiziert werden, die typischerweise implementiert werden sollten, um den Datenschutz und die Datensicherheit zu gewährleisten.

\begin{enumerate}
    
    \item \textbf{Datenschutz durch Technikgestaltung und durch datenschutzfreundliche Voreinstellungen:}
    Implementierung von Verschlüsselungstechnologien für Datenübertragungen (z.B. TLS/SSL) und für gespeicherte Daten. Begrenzung der Datenerhebung, -speicherung und -zugriffe auf das für die spezifischen Zwecke notwendige Minimum.
    Automatische Löschung oder Anonymisierung nicht mehr benötigter Daten.
    
    \item \textbf{Zugangskontrollen:}
    Einsatz von starken Authentifizierungsverfahren für Nutzerzugänge, einschließlich Multi-Faktor-Authentifizierung.
    
    \item \textbf{Datensicherheit:}
    Einsatz von Firewalls und anderen Netzwerksicherheitstechnologien zur Abwehr externer Angriffe. Regelmäßige Sicherheitsüberprüfungen und Penetrationstests, um Schwachstellen zu identifizieren und zu beheben.
    
    \item \textbf{Datenschutz-Folgenabschätzung und regelmäßige Überprüfung:}
    Durchführung von Datenschutz-Folgenabschätzungen für Verarbeitungstätigkeiten, die ein hohes Risiko für die Rechte und Freiheiten natürlicher Personen darstellen. Regelmäßige Überprüfung und Aktualisierung der Datenschutzpraktiken und -maßnahmen.
    
    \item \textbf{Schulung und Bewusstseinsbildung:}
    Schulung der Mitarbeiter in Datenschutzpraktiken und -richtlinien.
    Sensibilisierung für Datenschutzrisiken und für den richtigen Umgang mit personenbezogenen Daten.
    
    \item \textbf{Verfahren zur Reaktion auf Datenschutzverletzungen:}
    Etablierung eines Prozesses für die Meldung und Behebung von Datenschutzverletzungen, einschließlich der Benachrichtigung der zuständigen Aufsichtsbehörde und der betroffenen Personen.
\end{enumerate}
    
    Ob diese Maßnahmen ausreichen, hängt von einer kontinuierlichen Bewertung der Risiken und der Effektivität der implementierten Schutzmaßnahmen ab. Es ist wichtig, dass das Projektmanagement die sich ständig weiterentwickelnden Bedrohungen und Technologien im Auge behält und die Maßnahmen entsprechend anpasst. Zusätzliche Maßnahmen könnten notwendig sein, um neue Risiken zu adressieren oder um auf Feedback von Nutzern oder Änderungen in den rechtlichen Anforderungen zu reagieren.
\chapter{Verarbeitungsvorgänge personenbezogener Daten in der Plattform}

Die Verarbeitung personenbezogener Daten durch die Plattform für das persönliche Finanzmanagement umfasst verschiedene Vorgänge, die in bestimmten Situationen während des Ablaufs der IT-Anwendung stattfinden. Hier sind die konkreten Verarbeitungsvorgänge detailliert beschrieben:

\begin{enumerate}
    \item \textbf{Erhebung von Daten:}
        Beim Erstellen eines Kontos auf der Plattform werden persönliche Identifikationsinformationen wie Name und E-Mail-Adresse erhoben.
        Finanzielle Informationen werden erhoben, wenn Nutzer ihre Einnahmen, Ausgaben, Budgets und Investitionsdaten manuell eingeben oder wenn diese Informationen automatisch aus verknüpften Finanzkonten importiert werden.

    \item \textbf{Speicherung von Daten:}
        Alle erfassten Daten werden auf sicheren Servern gespeichert, um die Nutzung der Plattformfunktionen zu ermöglichen.
        Finanzielle Ziele und Fortschrittsdaten der Nutzer werden ebenfalls gespeichert, um die Verfolgung und Analyse des finanziellen Fortschritts zu ermöglichen.

    \item \textbf{Analyse von Daten:}
        Die Plattform nutzt KI-gestützte Technologien, um Budgetplanung und -analyse anzubieten. Dabei werden die finanziellen Daten der Nutzer analysiert, um personalisierte Empfehlungen und Einsichten zu bieten.
        Ausgabenmuster und Investitionen werden analysiert, um Spar- und Investitionsempfehlungen abzuleiten.

    \item \textbf{Veränderung von Daten:}
        Nutzer haben die Möglichkeit, ihre persönlichen und finanziellen Daten jederzeit zu aktualisieren oder zu korrigieren, um ihre finanzielle Situation genau widerzuspiegeln.

    \item \textbf{Löschung von Daten:}
        Auf Anfrage der Nutzer können persönliche Daten gelöscht werden, sofern keine gesetzlichen Aufbewahrungspflichten entgegenstehen. Daten werden auch gelöscht, wenn sie für die Zwecke, für die sie erhoben wurden, nicht mehr notwendig sind oder wenn das Nutzerkonto gelöscht wird.
\end{enumerate}

Jeder dieser Verarbeitungsvorgänge ist entscheidend für die Funktionalität der Plattform und muss unter strikter Einhaltung von Datenschutzgesetzen und -best practices durchgeführt werden. Besonders wichtig ist dabei die Gewährleistung der Datensicherheit und -integrität, die Verschlüsselung von Daten, der Schutz vor unbefugtem Zugriff und die transparente Kommunikation mit den Nutzern über die Nutzung ihrer Daten.
\chapter{Verarbeitung personenbezogener Daten unter Berücksichtigung von Art. 9 Abs. 1 DS-GVO}

Das Geschäftsmodell des vorgeschlagenen Projekts \enquote{Personal Finance Management} konzentriert sich auf die Entwicklung einer Plattform für das persönliche Finanzmanagement, die Einzelpersonen dabei unterstützt, ihre finanzielle Situation zu managen und ihre finanziellen Ziele zu erreichen. Zu den Hauptfunktionen gehören die Verfolgung von Einnahmen und Ausgaben, KI-gestützte Budgetplanung und -analyse, Spar- und Investitionsmanagement, Schuldenabbau sowie die Verfolgung finanzieller Ziele und Fortschritte.

Bei der Untersuchung, welche personenbezogenen Daten durch diese IT-Anwendung konkret verarbeitet werden, können folgende Datenpunkte identifiziert werden:

\begin{enumerate}
    \item \textbf{Personenbezogene Daten:} Um ein Konto einzurichten und die Plattform personalisiert nutzen zu können, werden wahrscheinlich Informationen wie Name, E-Mail-Adresse und möglicherweise auch Wohnort und Alter der Nutzer verarbeitet.
    \item \textbf{Finanzielle Informationen:} Dazu gehören Daten über Einnahmen, Ausgaben, Investitionen, Budgets und Schulden. Diese Daten werden benötigt, um den Nutzern eine umfassende Lösung für ihr Finanzmanagement anbieten zu können.
\end{enumerate}

Ob es sich dabei um sensible personenbezogene Daten im Sinne von Art. 9 Abs. 1 DS-GVO handelt, hängt davon ab, ob Daten verarbeitet werden, aus denen die rassische und ethnische Herkunft, politische Meinungen, religiöse oder weltanschauliche Überzeugungen oder die Gewerkschaftszugehörigkeit hervorgehen, sowie ob es sich um genetische Daten, biometrische Daten zur eindeutigen Identifizierung einer natürlichen Person, Gesundheitsdaten oder Daten zum Sexualleben oder der sexuellen Orientierung handelt.

Bei der primären Datenerhebung und -verarbeitung handelt es sich nicht unmittelbar um die Erhebung sensibler personenbezogener Daten im Sinne von Art. 9 DS-GVO. Die detaillierte Analyse finanzieller Transaktionen und Verhaltensweisen könnte jedoch indirekt sensible Informationen offenbaren, beispielsweise durch Rückschlüsse auf Gesundheitsausgaben, politische Spenden oder Mitgliedsbeiträge an Gewerkschaften und religiöse Gruppen. Daher ist es wichtig, dass bei der Entwicklung und dem Betrieb der Plattform strenge Datenschutzrichtlinien und Sicherheitsmaßnahmen umgesetzt werden, um die Privatsphäre der Nutzer zu schützen und die Einhaltung der DS-GVO zu gewährleisten.
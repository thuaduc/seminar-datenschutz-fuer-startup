\chapter{Rollen und Verantwortlichkeiten in der Datenverarbeitung der Plattform}

%TODO: rewrite this part
Basierend auf der Struktur DES Projekte und den Anforderungen der DSGVO können folgende Annahmen getroffen werden:

\begin{enumerate}
    \item \textbf{Verantwortlicher:}
    Der Verantwortliche ist die natürliche oder juristische Person, die allein oder gemeinsam mit anderen über die Zwecke und Mittel der Verarbeitung von personenbezogenen Daten entscheidet. Im Kontext des beschriebenen Projekts wäre dies typischerweise das Unternehmen oder die Organisation, die die Finanzmanagement-Plattform initiiert und betreibt. Sie sind verantwortlich für die Einhaltung der Datenschutzgesetze und müssen sicherstellen, dass alle Verarbeitungstätigkeiten der personenbezogenen Daten im Einklang mit der DSGVO stehen.

    \item \textbf{Auftragsverarbeiter:}
    Ein Auftragsverarbeiter ist eine natürliche oder juristische Person, die personenbezogene Daten im Auftrag des Verantwortlichen verarbeitet. In der Entwicklung und dem Betrieb einer solchen Plattform könnten externe Dienstleister als Auftragsverarbeiter eingesetzt werden, z.B. für Cloud-Hosting, Datenanalyse, Kundensupport oder Finanzdienstleistungen. Der Einsatz von Auftragsverarbeitern erfordert einen Vertrag oder eine andere rechtliche Grundlage, die den Auftragsverarbeiter zur Einhaltung der DSGVO und zum Schutz der personenbezogenen Daten verpflichtet.

    \item \textbf{Eigene Rolle als Auftragsverarbeiter:}
    Falls das Unternehmen oder die Organisation, die das Projekt durchführt, personenbezogene Daten im Auftrag eines anderen Unternehmens oder einer anderen Organisation verarbeitet (z.B. wenn es als Anbieter von Finanzmanagement-Lösungen für Banken oder andere Finanzinstitutionen fungiert), würde es selbst als Auftragsverarbeiter agieren. In diesem Fall müsste es die Anforderungen der DSGVO für Auftragsverarbeiter erfüllen, einschließlich der Verarbeitung von Daten gemäß den Anweisungen des Verantwortlichen und der Implementierung angemessener technischer und organisatorischer Maßnahmen zum Schutz der Daten.
\end{enumerate}


Es ist entscheidend, dass die Rollen und Verantwortlichkeiten im Zusammenhang mit der Datenverarbeitung klar definiert und dokumentiert werden, um die Einhaltung der Datenschutzgesetze zu gewährleisten. Die Verantwortlichen müssen auch die Transparenz gegenüber den betroffenen Personen sicherstellen, indem sie sie über die Verarbeitung ihrer Daten, die Identität des Verantwortlichen und gegebenenfalls des Auftragsverarbeiters informieren.
\chapter{Rollen und Verantwortlichkeiten in der Datenverarbeitung der Plattform}

Im Kontext der DS-GVO wird zwischen dem Verantwortlichen und dem Auftragsverarbeiter unterschieden. Gemäß Art. 28 DS-GVO ist der Verantwortliche die Person oder Stelle, die über die Zwecke und Mittel der Verarbeitung von personenbezogenen Daten entscheidet. Der Auftragsverarbeiter hingegen ist eine Person oder Stelle, die personenbezogene Daten im Auftrag des Verantwortlichen verarbeitet.

\begin{itemize}
    \item \textbf{Wer ist der Verantwortliche?}
    In Bezug auf das Projekt wäre der Verantwortliche das Unternehmen, die die Plattform betreibt. Dieses Unternehmen entscheidet über die Zwecke (z.B. die Bereitstellung von Dienstleistungen zur Finanzverwaltung) und die Mittel (z.B. die technischen Lösungen und Prozesse) der Datenverarbeitung.
    
    \item \textbf{Einsatz von Auftragsverarbeitern}
    Das Unternehmen könnte zur Unterstützung bestimmter Funktionen der Plattform oder zur Speicherung der Daten Auftragsverarbeiter einsetzen. Beispiele für Auftragsverarbeiter könnten Cloud-Dienstanbieter, Anbieter von Kundenbetreuungssoftware oder Dienstleister für die Datenanalyse sein.
    
    \item \textbf{Verantwortung bei Einsatz von Auftragsverarbeitern}
    Gemäß Art. 28 DS-GVO muss der Verantwortliche sicherstellen, dass Auftragsverarbeiter geeignete technische und organisatorische Maßnahmen ergreifen, um einen Datenschutz gemäß den Anforderungen der DS-GVO zu gewährleisten. Der Verantwortliche und der Auftragsverarbeiter müssen einen Vertrag abschließen, der den Gegenstand und die Dauer der Verarbeitung, die Art und den Zweck der Verarbeitung, die Art der personenbezogenen Daten, die Kategorien betroffener Personen und die Pflichten und Rechte des Verantwortlichen festlegt. 
    
    Wichtig ist auch, dass Auftragsverarbeiter nur auf dokumentierte Weisung des Verantwortlichen handeln dürfen und die personenbezogenen Daten nicht für eigene Zwecke verwenden dürfen. Verstößt ein Auftragsverarbeiter gegen diese Vorgaben, indem er über die Zwecke und Mittel der Verarbeitung entscheidet, gilt er gemäß der DS-GVO als Verantwortlicher in Bezug auf diese Verarbeitung. \\
    
\end{itemize}
\chapter{Verarbeitung personenbezogener Daten unter Berücksichtigung von Art. 9 Abs. 1 DS-GVO}

Das Geschäftsmodell der \enquote{Finance Management App} konzentriert sich auf die Entwicklung einer Plattform für das persönliche Finanzmanagement, die unterstützt Privatpersonen dabei, ihre finanzielle Situation zu managen und ihre finanziellen Ziele zu erreichen. Zu den Hauptfunktionen gehören die Verfolgung von Einnahmen und Ausgaben, KI-gestützte Budgetplanung und -analyse, Spar- und Investitionsmanagement sowie die Verfolgung finanzieller Ziele. Die Verarbeitung personenbezogener Daten ist für die Funktionen der Plattform, die es Einzelpersonen ermöglichen, ihre finanzielle Situation zu verwalten und finanzielle Ziele zu erreichen, von zentraler Bedeutung. Im Rahmen der Anwendung werden folgende Arten von Daten verarbeitet
\begin{enumerate}
    \item \textbf{Personenbezogene Daten:} 
    Um ein Konto einzurichten und die Plattform personalisiert nutzen zu können, werden Informationen wie Name, E-Mail-Adresse und Alter des Nutzers verarbeitet.
    \item \textbf{Finanzielle Informationen:} 
    Dazu gehören Daten über Einnahmen, Ausgaben, finanzielle Ziele und Plannen. Die Daten werden für die Kernfunktionen der Plattformdienste benötigt.
    \item \textbf{Nutzungsdaten und Interaktionsdaten mit der Plattform:} 
    Dazu gehören Daten darüber, welche Funktionen sie nutzen und wie sie mit der Plattform interagieren. Diese Informationen sind wichtig, um die Nutzererfahrung zu verbessern.

\end{enumerate}

Ob es sich um sensible personenbezogene Daten im Sinne von Art. 9 Abs. 1 DS-GVO handelt, hängt davon ab, ob Daten verarbeitet werden, aus denen die rassische und ethnische Herkunft, politische Meinungen, religiöse oder weltanschauliche Überzeugungen oder die Gewerkschaftszugehörigkeit hervorgehen, sowie ob es sich um genetische Daten, biometrische Daten zur eindeutigen Identifizierung einer natürlichen Person, Gesundheitsdaten oder Daten zum Sexualleben oder der sexuellen Orientierung handelt.

Bei der primären Datenerhebung und -verarbeitung handelt es sich nicht unmittelbar um die Erhebung sensibler personenbezogener Daten im Sinne von Art. 9 DS-GVO. Die detaillierte Analyse finanzieller Transaktionen und Verhaltensweisen könnte jedoch indirekt sensible Informationen offenbaren, beispielsweise durch Rückschlüsse auf Gesundheitsausgaben, politische Spenden oder Mitgliedsbeiträge an Gewerkschaften und religiöse Gruppen. Daher ist es wichtig, dass bei der Entwicklung und dem Betrieb der Plattform strenge Datenschutzrichtlinien und Sicherheitsmaßnahmen umgesetzt werden, um die Privatsphäre der Nutzer zu schützen und die Einhaltung der DS-GVO zu gewährleisten.